\documentclass[11pt,a4paper]{article}
\usepackage[utf8]{inputenc}
\usepackage[spanish,es-tabla]{babel}
\usepackage[margin=2cm]{geometry}
\usepackage{amsmath,amssymb}
\usepackage{booktabs}
\usepackage{hyperref}

\title{\textbf{Ejercicio 2: Cuantización}}
\author{Jordi Blasco Lozano}
\date{}

\begin{document}
\maketitle

\section*{Planteamiento}
Se considera la cuantización simétrica uniforme de un vector en punto flotante a enteros de 8 bits.
\[
x = [1.23,\,-0.87], \qquad n = 8.
\]

\subsection*{Datos iniciales}
\begin{itemize}
  \item Entrada: $x = [1.23,\,-0.87]$.
  \item Número de bits: $n = 8$.
  \item Valores cuantizados: $q \in \mathbb{Z}$.
\end{itemize}

\section*{Notación y fórmulas}
\begin{align*}
q_{\min} &= -2^{n-1}, &
q_{\max} &= 2^{n-1} - 1, \\
\alpha &= \max_i |x_i|, &
s &= \frac{\alpha}{q_{\max}}, \\
q_i &= \operatorname{round}\!\left(\frac{x_i}{s}\right), &
\hat{x}_i &= q_i\,s.
\end{align*}
El error cuadrático medio se define como:
\[
\mathrm{MSE} = \frac{1}{N}\sum_{i=1}^{N}(x_i-\hat{x}_i)^2.
\]

\section{Apartado 1: Cálculo de parámetros de cuantización}
\subsection*{Rango cuantizado}
\begin{align*}
q_{\min} &= -2^{8-1} = -2^7 = -128, \\
q_{\max} &= 2^{8-1}-1 = 2^7-1 = 127.
\end{align*}

\subsection*{Valor absoluto máximo}
\begin{align*}
\alpha &= \max(|1.23|,\;|-0.87|)
= \max(1.23,\;0.87)
= 1.23.
\end{align*}

\subsection*{Factor de escala}
\begin{align*}
s &= \frac{\alpha}{q_{\max}}
= \frac{1.23}{127}
\approx 0.00969.
\end{align*}

\section{Apartado 2: Cuantización y decuantización}
\subsection*{Cuantización de $x_1 = 1.23$}
\begin{align*}
q_1 &= \operatorname{round}\!\left(\frac{1.23}{0.00969}\right)
= \operatorname{round}(126.93\ldots)
= 127.
\end{align*}

\subsection*{Cuantización de $x_2 = -0.87$}
\begin{align*}
q_2 &= \operatorname{round}\!\left(\frac{-0.87}{0.00969}\right)
= \operatorname{round}(-89.78\ldots)
= -90.
\end{align*}

\subsection*{Vector cuantizado}
\[
q = [127,\,-90].
\]

\subsection*{Reconstrucción}
\begin{align*}
\hat{x}_1 &= 127\cdot 0.00969 = 1.2306, \\
\hat{x}_2 &= (-90)\cdot 0.00969 = -0.8721.
\end{align*}

\section{Apartado 3: Error de cuantización}
\subsection*{Error absoluto}
\begin{align*}
\epsilon_1 &= |1.23 - 1.2306| = 0.0006, \\
\epsilon_2 &= |-0.87 - (-0.8721)| = 0.0021.
\end{align*}

\subsection*{Error cuadrático medio}
\begin{align*}
\mathrm{MSE}
&= \frac{1}{2}\left[(0.0006)^2 + (0.0021)^2\right] \\
&= \frac{1}{2}\left[0.00000036 + 0.00000441\right] \\
&= \frac{0.00000477}{2}
\approx 2.39\times10^{-6}.
\end{align*}

\section*{Resumen final}
\begin{table}[h]
\centering
\begin{tabular*}{0.8\linewidth}{@{\extracolsep{\fill}}ccccc@{}}
\toprule
Elemento & Original ($x$) & Cuantizado ($q$) & Reconstruido ($\hat{x}$) & Error \\ \midrule
$x_1$ & 1.23 & 127 & 1.2306 & 0.0006 \\
$x_2$ & $-0.87$ & $-90$ & $-0.8721$ & 0.0021 \\ \bottomrule
\end{tabular*}
\end{table}

\end{document}


