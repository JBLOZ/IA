% !TeX program = xelatex
% !TeX TXS-program:compile = txs:///xelatex/[--shell-escape]
%%%%%%%%%%%%%%%%%%%%%%%%%%%%%%%%%%%%%%%%%%%%%%%%%%%%%%%%%%%%%%%%%%%%%%%%
% Plantilla TFG/TFM
% Escuela Politécnica Superior de la Universidad de Alicante
% Realizado por: Jose Manuel Requena Plens
% Contacto: info@jmrplens.com / Telegram:@jmrplens
%%%%%%%%%%%%%%%%%%%%%%%%%%%%%%%%%%%%%%%%%%%%%%%%%%%%%%%%%%%%%%%%%%%%%%%%

% Elige si deseas optimizar la ejecución del proyecto almacenando las figuras generadas con TikZ y PGF en una carpeta (archivos/figuras-procesadas).
% 1 - Si, 2 - No
\def\OptimizaTikZ{2}

% Archivo .TEX que incluye todas las configuraciones del documento y los paquetes. Añade todo aquello que necesites utilizar en el documento en este archivo.
% En él se encuentra la configuración de los márgenes, establecidos según las directrices de estilo de la EPS.
\input{include/configuracioninicial}

% Obligatorio colocar en el main en la version para Overleaf, no eliminar
\if\OptimizaTikZ 1
\tikzexternalize[prefix=archivos/figuras-procesadas/] % Ruta
\fi 

%%%%%%%%%%%%%%%%%%%%%%%%%%%%%%%%%%%%%%%%%%%%%%%%%%%%%%%%%%%%%%%%%%%%%%
% INFORMACIÓN DEL TFG
% Comentar lo que NO se desee añadir y sustituir con la información correcta.
%%%%%%%%%%%%%%%%%%%%%%%%%%%%%%%%%%%%%%%%%%%%%%%%%%%%%%%%%%%%%%%%%%%%%%
% Título y subtítulo
\newcommand{\titulo}{Título del Trabajo Fin de Grado/Máster}
\newcommand{\subtitulo}{Subtítulo del proyecto}
% Datos del autor
\newcommand{\miNombre}{Nombre Apellido1 Apellido2 (alumno)}
\newcommand{\miEmail}{nombre@alu.ua.es}
% Datos del tutor/es
\newcommand{\miTutor}{Nombre Apellido1 Apellido2 (tutor1)}
\newcommand{\miTutorB}{Nombre Apellido1 Apellido2 (tutor2)}
\newcommand{\departamentoTutor}{Departamento del tutor}
\newcommand{\departamentoTutorB}{Departamento del cotutor}
% Datos de la facultada y universidad
\newcommand{\miFacultad}{Escuela Politécnica Superior}
\newcommand{\miFacultadCorto}{EPS UA}
\newcommand{\miUniversidad}{\protect{Universidad de Alicante}}
\newcommand{\miUbicacion}{Alicante}

%%%%%%%%%%%%%%%%%%%%%%%%%%%%%%%%%%%%%%%%%%%%%%%%%%%%%%%%%%%%%%%%%%%%%%
% INDICA TU TITULACIÓN
% ID	GRADO -------------------------------------------------
% 1		Ingeniería en Imagen y Sonido en Telecomunicación
% 2		Ingeniería Civil
% 3		Ingeniería Química
% 4		Ingeniería Informática
% 5		Ingeniería Multimedia
% 6		Arquitectura Técnica
% 7		Arquitectura
% 8		Robótica
% %		%%%%%%%%%%%%
% ID	MÁSTER ------------------------------------------------
% A		Telecomunicación
% B		Caminos, Canales y Puertos
% C		Gestión en la Edificación
% D		Desarrollo Web
% E		Materiales, Agua, Terreno
% F		Informática
% G 	Automática y Robótica
% H		Prevención de riesgos laborales
% I		Gestión Sostenible Agua
% J		Desarrollo Aplicaciones Móviles
% K		Ingeniería Química
% L		Ciberseguridad
%%%%%%%%%%%%%%%%%%%%%%%%%%%%%%%%%%%%%%%%%%%%%%%%%%%%%%%%%%%%%%%%%%%%%%%%%
%!!!!!!!!!!!!!!!!!!!!!!!!!!!!!!!!!!!!!!!!!!!!!!!!!!!!!!!!!!!!!!!!!!!!!%%%
																		%
\def\IDtitulo{1} % INTRODUCE LA ID DE TU TITULACIÓN						%
																		%
%!!!!!!!!!!!!!!!!!!!!!!!!!!!!!!!!!!!!!!!!!!!!!!!!!!!!!!!!!!!!!!!!!!!!!%%%
%%%%%%%%%%%%%%%%%%%%%%%%%%%%%%%%%%%%%%%%%%%%%%%%%%%%%%%%%%%%%%%%%%%%%%%%%

% Configuración automática según el identificador elegido
\input{include/configuraciontitulacion} 

% Información añadida a las propiedades del archivo PDF.
\hypersetup{
pdfauthor = {\miNombre~(\miEmail)},
pdftitle = {\titulo},
}

%%
% Archivo de acrónimos
%%
\makeglossaries % Genera la base de datos de acrónimos
\input{anexos/acronimos.tex} % Archivo que contiene los acrónimos

%%%%%%%%%%%%%%%%%%%%%%%% 
% INICIO DEL DOCUMENTO
% A partir de aquí debes empezar a realizar tu TFG/TFM
%%%%%%%%%%%%%%%%%%%%%%%%
\begin{document}

% Números romanos hasta el mainmatter.
\frontmatter

% PORTADA
\input{include/portada/portada_color} % Portada Color
\input{include/portada/portada_bn} % Portada B/N

%%%%% PREAMBULO
% Incluye el .tex que contiene el preámbulo, agradecimientos y dedicatorias.
\input{capitulos/preliminaresconagradecimientos} 

% Incluye después del archivo anterior el indice y lista de figuras, tablas y códigos.
\tableofcontents	% Índice
\listoffigures		% Índice de figuras
\listoftables		% Índice de tablas
\lstlistoflistings	% Índice de códigos

% Inicia la numeración habitual.
\mainmatter

%%%%
% CONTENIDO. CAPÍTULOS DEL TRABAJO - Añade o elimina según tus necesidades
%%%%
\input{capitulos/Introduccion}	% Plantilla: Se muestran contenidos
\input{capitulos/marcoteorico}	% Plantilla: Se muestran listas
\input{capitulos/objetivos}		% Plantilla: Se muestran tablas
\input{capitulos/metodologia}	% Plantilla: Se muestran figuras
\input{capitulos/desarrollo}		% Plantilla: Se muestran listados
\input{capitulos/resultados}		% Plantilla: Se muestran gráficas
\input{capitulos/conclusiones}	% Plantilla: Se muestran matemáticas

%%%%
% CONTENIDO. BIBLIOGRAFÍA.
%%%%
\nocite{*} %incluye TODOS los documentos de la base de datos bibliográfica sean o no citados en el texto
\bibliography{bibliografia/bibliografia} % Archivo que contiene la bibliografía
\bibliographystyle{apacite}

%%%%
% CONTENIDO. LISTA DE ACRÓNIMOS. Comenta las líneas si no lo deseas incluir.
%%%%
% Incluye el listado de acrónimos utilizados en el trabajo. 
\printglossary[style=modsuper,type=\acronymtype,title={Lista de Acrónimos y Abreviaturas}]
% Añade el resto de acrónimos si así se desea. Si no elimina el comando siguiente
\glsaddallunused 

%%%%
% CONTENIDO. Anexos - Añade o elimina según tus necesidades
%%%%
\appendix % Inicio de los apéndices
\input{anexos/anexo_I}
\input{anexos/anexo_2}
\input{anexos/anexo_3}

\end{document}