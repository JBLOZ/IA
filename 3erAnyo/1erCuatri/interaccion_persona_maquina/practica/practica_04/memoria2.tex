Objetivo y alcance
El objetivo es habilitar una interacción natural vía Telegram para solicitar análisis de sentimiento de una acción o ETF y devolver una recomendación a medio plazo, implementando el flujo en n8n con un LLM y al menos dos aplicaciones integradas.​​
El alcance incluye: captura de mensajes en Telegram, filtrado de comandos, extracción del ticker, recuperación de noticias con la Search API de Perplexity y síntesis de un informe con recomendación, más un paso de confirmación con el usuario.​​

Requisitos del seminario
La solución incorpora un LLM que procesa las instrucciones del usuario y genera respuestas cuando corresponde, cumpliendo el requisito de uso de modelos de lenguaje.​​
Integra al menos dos aplicaciones conectadas: Telegram (entrada/salida) y Perplexity Search API (recuperación de noticias), asegurando la integración pedida.​​
El diseño multiagente define un mínimo de tres agentes que interactúan con el LLM: filtro/validación de input, recuperación de evidencias y análisis con recomendación, además de un agente de confirmación.​​
Existe una interacción de confirmación con el usuario que solicita autorización antes de devolver y registrar la recomendación final, cumpliendo el requisito de confirmación.​​

Arquitectura de alto nivel
Interfaz de entrada y salida: Telegram sirve como canal conversacional para recibir comandos y enviar resultados, soportado por los nodos Trigger y Send Message de n8n.​
Orquestación: n8n coordina el pipeline con nodos de disparo, lógica de filtrado, extracción, peticiones HTTP y mensajería de retorno.​
Recuperación de información: la Perplexity Search API recibe la consulta “Last news about the stock of <ticker>” y devuelve resultados clasificados y preparados para consumo.​
Despliegue y webhook: Cloudflare Tunnel expone de forma segura el endpoint local de n8n para que Telegram pueda entregar webhooks, con URL pública dinámica en quick tunnels.​

Diseño multiagente
Agente 1 — Ingesta y filtro: capta mensajes con el Telegram Trigger y restringe el flujo a entradas con el comando /analysis, apoyándose en la capacidad del nodo para recibir eventos de tipo Message.​
Agente 2 — Extracción del ticker: un bloque de código (Code node) aísla el texto posterior a /analysis y normaliza el ticker, produciendo un JSON compacto con “text” y “chat_id” para consumo posterior.​
Agente 3 — Recuperación de evidencias: un HTTP Request invoca la Perplexity Search API con POST /search para obtener fragmentos y documentos relevantes, limitando longitud por item.​
Agente 4 — Analista LLM: un paso de LLM sintetiza sentimiento de mercado y propone una acción a medio plazo entre compra fuerte, compra, holdear, vender o venta fuerte a partir de las evidencias.​
Agente 5 — Confirmación y entrega: antes de enviar y registrar la recomendación final, se pide confirmación al usuario y se utiliza la capacidad de respuesta de Telegram para mensajería/teclados.​

Flujo paso a paso
Recepción: Telegram Trigger activa el flujo al recibir un nuevo mensaje, exponiendo el payload con el texto del chat.​
Filtrado: se descartan mensajes que no contengan “/analysis”, conservando únicamente las solicitudes válidas para análisis.​
Extracción: un nodo de código devuelve sólo lo que sigue a “/analysis”, mapeándolo a { text: “TICKER”, chat_id } para el resto del pipeline.​
Búsqueda: HTTP Request a la Perplexity Search API (POST /search) con la query “Last news about the stock of {{ $json.text }}”, solicitando un conjunto finito de resultados.​
Síntesis: el LLM consume dichas noticias y produce un resumen con sentimiento y recomendación justificada.​
Confirmación: se solicita confirmación al usuario antes de publicar y registrar la salida recomendada, gestionando la interacción de forma nativa con Telegram.​
Entrega: tras la confirmación, Telegram envía el mensaje final al chat_id original con el texto formateado.​

Nodos y componentes en n8n
Telegram Trigger: configura el evento “Message” para escuchar mensajes entrantes del bot y activar el flujo.​
Lógica de filtrado: una condición/IF restringe a mensajes que contengan “/analysis” para mejorar precisión y coste.​
Code (JavaScript): extrae el texto después de “/analysis” y devuelve el JSON mínimo con el ticker y chat_id.​
HTTP Request — Perplexity Search API: realiza POST /search con el API key, controlando número de items y longitud de contexto.​
LLM step: genera el análisis y la recomendación basándose en los resultados recuperados.​
Telegram Send Message: envía borrador para revisión y mensaje final confirmado al usuario original.​

Integración con Perplexity Search API
La API de búsqueda expone un endpoint de tipo POST que devuelve resultados clasificados con opciones de filtrado y personalización.​
Este diseño utiliza la consulta parametrizada con el ticker, limitando la cantidad y tamaño de resultados para mantener la latencia y la claridad del prompt.​
La salida de búsqueda alimenta al LLM como contexto para un análisis con trazabilidad a evidencias recientes.​

Telegram: captura y respuesta
El Telegram Trigger de n8n recibe eventos de mensaje, permitiendo arrancar el flujo y acceder al contenido y metadatos del chat.​
El Telegram node soporta operaciones de envío de mensajes y respuestas a interacciones, útiles para confirmar y comunicar resultados.​
Esta combinación habilita un circuito conversacional cerrado: entrada del usuario, procesamiento, confirmación y salida.​

Despliegue y webhook con Cloudflare Tunnel
Telegram necesita acceder a un endpoint público HTTPS para entregar webhooks, por lo que se expone el servidor local mediante Cloudflare Tunnel.​
Con quick tunnels se puede publicar localhost con un comando “cloudflared tunnel --url http://localhost:<puerto>” y obtener una URL pública temporal.​
Es habitual automatizar la lectura de esa URL y propagarla a variables de entorno para actualizar la configuración del webhook sin intervención manual.​